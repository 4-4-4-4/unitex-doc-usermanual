\chapter*{Introduction}
\addcontentsline{toc}{chapter}{Introduction}

Unitex est un ensemble de logiciels permettant de traiter des textes en langues naturelles en
utilisant des ressources linguistiques. Ces ressources se présentent sous la forme de dictionnaires
électroniques, de grammaires et de tables de lexique-grammaire. Elles sont issues de travaux initiés
sur le français par Maurice Gross au Laboratoire d’Automatique Documentaire et Linguistique (LADL)
. \index{LADL} Ces travaux ont été étendus à d’autres langues au travers du réseau de laboratoires
RELEX.

\bigskip
\noindent Les dictionnaires électroniques décrivent les mots simples et composés d’une langue en
leur associant un lemme ainsi qu’une série de codes grammaticaux, sémantiques et flexionnels. La présence de ces dictionnaires constitue une différence majeure par rapport aux outils
usuels de recherche de motifs, car on peut faire référence aux informations qu’ils contiennent
et ainsi décrire de larges classes de mots avec des motifs très simples. Ces dictionnaires sont
représentés selon le formalisme DELA et ont été élaborés par des équipes de linguistes pour
plusieurs langues (français, anglais, grec, italien, espagnol, allemand, thaï, coréen, polonais,
norvégien, portugais, etc...).


\bigskip
\noindent Les grammaires sont des représentations de phénomènes linguistiques par réseaux de
transitions récursifs (RTN), un formalisme proche de celui des automates à états finis. De
nombreuses études ont mis en évidence l’adéquation des automates aux problèmes linguistiques et ce,
aussi bien en morphologie qu’en syntaxe ou en phonétique. Les grammaires manipulées par Unitex
reprennent ce principe, tout en reposant sur un formalisme encore plus puissant que les automates.
Ces grammaires sont représentées au moyen de graphes que l’utilisateur peut aisément créer et
mettre à jour.

\bigskip
\noindent Les tables de lexique-grammaire sont des matrices décrivant les propriétés de certains
mots. De telles tables ont été élaborées pour tous les verbes simples du français dont elles
décrivent les propriétés syntaxiques. L’expérience ayant montré que chaque mot a un comportement
quasi unique, ces tables permettent de donner la grammaire de chaque élément de lexique, d’où le nom
de lexique-grammaire. Unitex permet de construire des grammaires à partir de telles tables.

\bigskip
\noindent Unitex est un moteur permettant d’exploiter ces ressources linguistiques. Ses
caractéristiques techniques sont la portabilité, la modularité, la possibilité de gérer des langues
possédant des systèmes d’écritures particuliers comme certaines langues asiatiques et l’ouverture,
grâce à une distribution en logiciel libre. Ses caractéristiques linguistiques sont celles
qui ont motivé l’élaboration des ressources : la précision, l’exhaustivité et la prise en compte
des phénomènes de figement, notamment en ce qui concerne le recensement des mots com-
posés.




\section*{Quoi de neuf depuis la version 3.0 ?}
\addcontentsline{toc}{section}{Quoi de neuf depuis la version 3.0 ?}
Voici les principales nouvelles fonctionnalités:
\begin{itemize}

  \item Moteur plus rapide qui utilise moins de pile.
  \item Version améliorée de \verb$CasSys$~:
  nouveaux fichiers \verb$csc$,
  ouverture de cascade possible aussi avec le menu \verb$FSGraph$,
  suppression du répertoire \verb$Share$,
  application de graphe jusqu'au point fixe, graphes génériques,
  redénormalisation du dernier fichier
  (chapitre \ref{chap-cassys}).
  \item Introduction du malgache.
  \item Publication de \verb$main_UnitexTool_C$ comme API publique.
  \item Version améliorée de l'éditeur de graphes~: sélection et édition des boites, ouverture, sauvegarde, exportation comme image (\ref{section-editing-graphs}, \ref{exporting-graphs}).
  \item Les commandes non applicables des menus sont maintenant grisées.
  \item Introduction de l'opérateur \verb$<n.LEMMA>$ pour la flexion en mode sémitique (non encore documenté).
  \item Introduction d'une liste de graphes et corpus récemment ouverts.
  \item Introduction d'une liste de fenêtres ouvertes.
  \item Compatibilité améliorée avec Ruby.
  \item Introduction de \verb$InstallLingResourcePackage$, un outil qui installe un paquetage de ressources et de scripts dans un environnement cible (non encore documenté).
  \item Introduction de \verb$RunScript$, qui exécute dans l'environnement cible des scripts installés par \verb$InstallLingResourcePackage$ (non encore documenté). Avec ces deux outils, on peut mettre au point des opérations Unitex dans un environnement et les déployer dans un autre.
  \item Introduction de l'option "match word boundaries" dans l'algorithme de recherche par intersection d'automates (\ref{section-locate-tfst})~: avec cette option, active par défaut pour la plupart des langues, \textit{enlever} and \textit{en lever} ne matchent pas (non encore documenté).
  \item Suivi amélioré des offsets, c'est-à-dire de la différence entre les adresses d'une même position dans un corpus suivant différentes versions du corpus (\ref{section-DumpOffsets}).
  \item Compilation quotidienne d'exécutables pour Windows (32-bit, 64-bit), GNU/Linux (Intel, Intel 64-bit) et 0S X (10.7+).
  \item Installateurs automatiques pour toutes ces plates-formes.

\end{itemize}

\bigskip
\noindent IMPORTANT: certains formats de fichiers ayant été changés et de nouveaux ayant été
ajoutés, nous vous recommandons d'effectuer un nouveau prétraitement de vos textes en particulier si
vous utilisez l'automate du texte.

\clearpage

\section*{Contenu}
\addcontentsline{toc}{section}{Content}
\noindent Le chapitre \ref{chap-install} décrit comment installer et lancer Unitex.

\bigskip \noindent Le chapitre \ref{chap-text} présente les différentes étapes de l'analyse d'un
texte.

\bigskip \noindent Le chapitre \ref{chap-dictionaries} décrit le formalisme 
des dictionnaires électroniques DELA et les différentes opérations qui peuvent leur être appliqués.

\bigskip \noindent Les chapitres \ref{chap-regexp} et \ref{chap-grammars}
présentent les différents moyens d’effectuer des recherches de motifs dans des textes.
Le chapitre \ref{chap-grammars} décrit en détail l'utilisation de l'éditeur de graphe.

\bigskip \noindent Le chapitre \ref{chap-advanced-grammars} est consacré aux différentes
utilisations possibles des grammaires. Les particularités de chaque type de grammaires y sont
présentées.

\bigskip \noindent Le chapitre \ref{chap-text-automaton} présente le concept d'automate du texte 
et décrit les propriétés de cette notion. Ce chapitre  décrit également les opérations sur cet
objet, en particulier, comment désambiguiser les items lexicaux avec le programme ELAG.

\bigskip \noindent Le chapitre \ref{chap-lexicon-grammar} contient une présentation des tables du
lexique-grammaire, et la description d'une méthode de construction de  grammaires fondées sur ces
tables.

\bigskip \noindent Le chapitre \ref{chap-alignment} décrit le module d'alignement de texte basé sur
l'outil XAlign.

\bigskip \noindent Le chapitre \ref{chap-multiflex} décrit le module de flexion des mots composés,
en tant que complément du système de flexion des mots simples, présenté au chapitre
\ref{chap-dictionaries}.

\bigskip \noindent Le chapitre \ref{chap-cassys} décrit le système de cascade de transducteur
CasSys.

\bigskip \noindent Le chapitre \ref{chap-external-programs} contient une description détaillée des
programmes externes qui composent le système Unitex.

\bigskip \noindent Le chapitre \ref{chap-file-formats} contient une description de tous les formats
de fichiers utilisés par Unitex.


\bigskip \noindent Le lecteur trouvera en annexe la licence LGPL sous  laquelle le code source
Unitex est diffusé, ainsi que la licence LGPLLR qui s'applique pour les données linguistiques
distribuées avec Unitex. Il y trouvera aussi la licence 2-clause BSD qui s'applique à la
bibliothèque TRE, utilisée par Unitex pour les filtres morphologiques.

\clearpage

\section*{Contributions à Unitex}
\addcontentsline{toc}{section}{Unitex contributeurs}
Unitex est né comme un pari sur la puissance de la philosophie Open Source dans le monde
universitaire (voir \url{http://igm.univ-mlv.fr/~unitex/why_unitex.html}), en s'appuyant sur l'hypothèse que les gens seraient intéressés à partager leurs connaissances et leurs compétences dans un tel projet ouvert.
%The following list sounds like Open Source is good for science:

\begin{itemize}                   
    \item Olivier Blanc: a intégré le système ELAG à Unitex, originellement conçu par Eric Laporte,
    Anne Monceaux et certains de leurs étudiants, a également écrit \verb+RebuildTfst+ (anciennement
     appelé \verb+MergeTextAutomaton+)
    \item Matthieu Constant: auteur de \verb+Grf2Fst2+
    \item Julien Decreton: auteur de l'éditeur de texte intégré à Unitex,
    	    a aussi réalisé la fonctionnalité \verb+undo+ de l'éditeur de graphe
    \item Claude Devis: ajout des filtres morphologiques, fondé sur la bibliothèque TRE
    \item Hyun-Gue Huh: auteur de l'outil de génération de dictionnaires coréens
    \item Claude Martineau: a travaillé sur la flexion des mots simples dans \verb+MultiFlex+
    \item Sebastian Nagel: a optimisé de nombreuses parties du code, il a également adapté
    	    \verb+PolyLex+ pour l'allemand et le russe
    \item Alexis Neme: a optimisé \verb+Dico+ et \verb+Tokenize+, a aussi intégré \verb+Locate+ dans \verb+Dico+ pour accepter des graphes dictionnaires
     \item Aljosa Obuljen: auteur de \verb+Stats+
     \item Sébastien Paumier: développeur principal
     \item Agata Savary: auteure de \verb+MultiFlex+
    \item Gilles Vollant: auteur de \verb+UnitexTool+, a optimisé beaucoup
    	    d'aspects du code d'Unitex (mémoire, vitesse, compatibilité multi-compilateur, etc)
    \item Patrick Watrin: auteur de \verb+XMLizer+, a travaillé sur l'intégration de \verb+XAlign+ à Unitex
    \item Anthony Sigogne: auteur de \verb+Tagger+ et de \verb+TrainingTagger+
    \item Nathalie Friburger: auteure de \verb+CaSsys+
\end{itemize}

\bigskip
\noindent Il faut ajouter que Unitex serait inutile sans les précieuses ressources linguistiques
qu'il renferme. Toutes ces ressources sont le fruit d'un énorme et difficile travail effectué par
des personnes qui ne doivent pas être oubliées. Certaines sont citées dans les avertissements qui
sont fournis avec les dictionnaires, une information complète est disponible sur:

\bigskip
\noindent \url{http://igm.univ-mlv.fr/~unitex/linguistic_data_bib.html}


\section*{Si vous utilisez Unitex dans des projets de recherche...}
\addcontentsline{toc}{section}{Si vous utilisez Unitex dans des projets de recherche...}
Unitex a été utlisé dans plusieurs projets de recherche. Certains sont listés dans la section 
``Related works'' de la page d'accueil d'Unitex. Si vous avez effectuer des travaux de recherche
avec Unitex (ressources, projet, article, thèse, ...) et si vous désirez qu'ils soient référencés
sur le site envoyez un mail à \url{unitex-devel@univ-mlv.fr}.

